\documentclass[12pt,a4paper,english,brazil]{abntex2}

\usepackage{babel}
\usepackage[utf8]{inputenc}
\usepackage[T1]{fontenc}
\usepackage{indentfirst}
\usepackage{verbatim}
\usepackage{graphicx}
\usepackage{microtype}
\usepackage{lipsum} % Apenas para gerar texto de exemplo
\usepackage{cite}
\usepackage{url}
\usepackage{graphicx}

\titulo{Tarefa 01}
\autor{Gabriel Costa Fassarella}
\data{12/08/2023} % Insira a data aqui
\posttitle{\par\end{center}\begin{center}\large UENF-CCT-LCMAT - Curso de Ciencia da Computacao \\ Análise e Projeto de Sistemas \\ Prof. Ausberto S. Castro Vera\end{center}\vskip0.5em}

\begin{document}
	\begin{figure}
		\centering
		\includegraphics[width=0.7\linewidth]{3-uenf_horizontal}
		\caption{}
		\label{fig:3-uenfhorizontal}
	\end{figure}

	\maketitle

	\section{Dados}
	
	Segundo Setzer(2002), dado é um símbolo que pode ser quantificado, ou seja, assumir um valor. Dessa forma, é possível concluir que um número pode ser um dado, assim como uma letra, visto caracteres podem assumir valores de uma base numérica. Logo, fotos, imagens, áudios e vídeos também podem ser dados, já que podem ser quantificados. Com isso, pode-se concluir que um dado é uma forma matemática que pode ser armazenada em um computador e ainda processada pelo mesmo.
	
	Dados são elementos extremamente fundamentais na computação, representando fatos, observações e descrições de todos os tipos que possa ser coletada e registrada, ou seja, um dado é uma representação bruta de um fato qualquer, podendo assim representar uma informação.
	
	De acordo com Setzer(2002), informação é uma abstração informal, ou seja, algo que pode ser formalizado por meio de uma teoria lógica ou matemática, por exemplo, "o Rio de Janeiro é lindo" é uma informação. Ou seja, uma informação é um conjunto de dados brutos que foram processados, organizados e interpretados fazendo com que tenham sentido dentro de um contexto ou propósito imposto.
	
	\section{Sistemas de Informação}
	
	Segundo Mattos(2017): "Um sistema é constituído de dois elementos: uma coleção de objetos, por um lado, e uma relação lógica entre eles, por outro.". Desse forma, um sistema empresarial possui os responsáveis por emitir os dados e aqueles que devem recepta-los, esses devem estar conectados por canais de comunicação de acordo com a lógica da empresa. Importante perceber que não é necessário utilizar de nenhum computador para o funcionamento desse sistema.
	
	De maneira simples, um sistema de informação é uma combinação organizada de indivíduos, processamentos, logística e tecnologia que trabalham em conjunto com o objetivo de coletar, distribuir, processar e interpretar dados, com o objetivo principal de melhorar a eficiência, eficácia e a competitividade da organização se utilizando de uma gestão otimizada dos dados.
	
	\section{Sistema Mais Antigo}
	
	De acordo com evidências históricas, e considerando sistemas com o objetivo de organizar, armazenar e distribuir informações, as formas de escritas das civilizações antigas como os hieróglifos egípcios por exemplo, junto do aperfeiçoamento da escrita com o surgimento dos escribas, foram os primeiros sistemas de informações. Além disso, as prensas móveis inventadas inicialmente na China em 1041 e aperfeiçoadas por Gutemberg por volta de 1439, tornou-se uma das primeiras grandes evoluções tecnológicas dos sistemas de informação.
	
	No contexto moderno de tecnológica de informação e da computação, os primeiros sistemas de informação começaram a surgir por volta da década de 1950 e 1960 (antes já haviam sistemas porém sem computadores), com o desenvolvimento dos computadores e do processamento dos dados, com isso a evolução dessa área passou a acompanhar a evolução da computação e crescer cada vez mais.\\  
	
	\section{Referências}

	Setzer, Valdemar W. "Dado, Informação, Conhecimento e Competência." In: Anais da Conferência Internacional de Tecnologias de Informação e Comunicação na Educação (TIC-EDUCA), São Paulo, SP, Brasil, 2002.\\

	MATTOS, Antonio Carlos M. Sistemas de Informação. Editora Saraiva Uni, 2017.\\
	
	PLANEZ, Paulo. Um pouco de História para entender os sistemas de informação. TIE Especialistas, 2023. Disponível em: \url{https://www.tiespecialistas.com.br/um-pouco-de-historia-para-entender-os-sistemas-de-informacao/.} Acesso em: 12 ago. 2023.
	
\end{document}
